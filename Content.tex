\begin{abstract}
	In this document we are going to prove \textbf{De Rham} theorem,
	which states that De Rham and singular cohomology,
	over a \textit{paracompact} manifold $M$, are isomorphic.
	Moreover we will prove the explicit form of this isomorphism,
	given by integration of forms on singular simplices.
	This result is going to be proved using $F$-injective resolutions and tools from sheaf theory.
\end{abstract} 

\section{Useful results}
Let $M$ be a \textit{paracompact smooth real manifold}.
\begin{lem}
	Every \textbf{fine} sheaf on $M$ is \textbf{soft}.
\end{lem} 
\begin{lem}\label{lem:gammainjectives}
	The full subcategory of \textbf{soft sheaves} is $\Gamma(M,-)$-injective.
\end{lem} 

\section{Singular Cohomology}
For this section consider $K$ to be a PID, $M$ a \textit{paracompact manifold} and $U \subset M$ an open subset of $M$.
Let us define the \textit{continuous singular simplices}:

\begin{defn}[Standard $p$-simplex]
	Let $\N \ni p \geq 1$. We define the \textit{standard p-simplex}
	\begin{equation}
		\Delta^p := \left\{ \left( a_1, \ldots, a_p \right) \in \R^p \ \middle|\ \sum_{i=1}^{p} a_i \leq 1 \text{ and, } \,\forall\, i \in \{1, \ldots, p \},\ a_i \geq 0  \right\}
	.\end{equation} 
	For $p = 0$ we set $\Delta^0 := \left\{ 0 \right\}$ the $1$-point space, and we call $\Delta^0$ the \textit{standard 0-simplex}.
\end{defn}

\begin{defn}[Continuous/differentiable singular p-simplex]
	We define a \textit{continuous singular simplex} $\sigma$ in $U$ to be a \textit{continuous} map $\sigma: \Delta^p \to U$.

	If $p \geq 1$ we say that $\sigma$ is a \textit{differentiable singular p-simplex} in $U$ iff it can be extended to a differentiable ($\mathscr{C}^\infty$) map of a neighborhood of $\Delta^p$ in $\R^p$ into $U$.
\end{defn}

\begin{rem}
	The theory can be developed also for \textit{differentiable} singular simplices, but it requires some care we are not willing to give.
	In general it is exactly the same as for \textit{continuous} ones, apart from where noted otherwise.
	In case you want to develop such theory, please refer to \cite{warner}.

	From now on we are going to deal only with \textit{continuous} singular simplices, hence we will stop indicating their continuity.
\end{rem}

\begin{defn}[Singular $p$-chains with integer coefficients]
Fixed $U \stackrel{\text{open}}{\subset} M$, we define $S_p(U)$ to be the \textit{free abelian group} (equivalently the \textit{free $\Z$-module}) generated by the singular $p$-simplices in $U$.
	Its elements are called \textit{singular $p$-chains with integer coefficients}, and can be written as finite formal sums, with integer coefficients, of singular simplices, such as:
	\begin{equation}
	c = \sum_{j=1}^{n} n_j \sigma_j
	,\end{equation} 
	where $n_j \in \Z\setminus \left\{ 0 \right\}$ and $\sigma_j$ a singular $p$-simplex for every $1 \leq j \leq n$.
\end{defn}


\begin{defn}[Boundary]
	We define, for each $p \geq 0$ and $0 \leq i \leq p+1$, the collection of maps $k_i^p: \Delta^p \to \Delta^{p+1}$:
	\begin{align}
	\begin{matrix}
		\text{for } p = 0,\quad 
		&\begin{dcases}
			k^0_0(0) := 1\\
			k^0_1(0) := 0\\
		\end{dcases} \\
		\\
		\text{for } p \geq 1,\quad 
		&\begin{cases}
		k_0^p(a_1, \ldots, a_p) := \left( 1 - \sum_{j=1}^{p} a_j, a_1, \ldots, a_p \right)\\
		k_i^p(a_1, \ldots, a_p) := \left( a_1, \ldots, a_{i-1}, 0, a_{i+1}, \ldots, a_p \right) \quad \text{for } 1 \leq i \leq p
	\end{cases} 
	\end{matrix} 
	.\end{align} 
	Given a singular $p$-simplex $\sigma$ in $U \stackrel{\text{open}}{\subset} M$, we define its $i$th \textit{face}, for $0 \leq i \leq p$, to be the singular ($p-1$)-simplex
	\begin{equation}
		\sigma^i := \sigma \circ k_i^{p-1} 
	.\end{equation}
	We will use the superscript index to denote the face of a singular simplex.
	Finally we define the \textit{boundary} of $\sigma$ to be the singular ($p-1$)-chain
	\begin{equation}
		\partial \sigma := \sum_{j=0}^{p} (-1)^j \sigma^j \in S_{p-1}(U)
	,\end{equation} 
	in which, as stated before, $\sigma^j$ denotes the $j$ th face of $\sigma$.

	Extending the \textit{boundary operator} $\partial$ by linearity, we obtain a homomorphism
	\begin{equation}
		\partial: S_p(U) \to S_{p-1}(U)
	.\end{equation} 
	More explicitly, the boundary operator acts as follows on a singular $p$-chain:
	\begin{equation}
		\partial \left( \sum_{j=1}^{n} a_j \sigma_j \right) := \sum_{j=1}^{n} a_j \partial \sigma_j = \sum_{j=1}^{n} \sum_{i=0}^{p} (-1)^i a_j \sigma_j^i
	.\end{equation} 
\end{defn}

\begin{lem}
	$k_i^{p+1} \circ k_j^p = k_{j+1}^{p+1} \circ k_{i}^p$ for any $p \geq 0$ and $i \leq j$.
\end{lem} 
\begin{proof}
	If $p = 0$ it can be checked directly (there are only 3 cases).
	For $p \geq 1$ it will be computed directly.
	The first term acts as
	\begin{equation}
		k_i^{p+1} \circ k_j^p (a_1, \ldots, a_p) =
	\begin{cases}
		\left( a_1, \ldots, a_{i-1}, 0, a_i, \ldots, a_{j-1}, 0, a_j, \ldots, a_p \right) & \text{ if } 1 \leq i < j\\
		\left( a_1, \ldots, a_{i-1}, 0, 0, a_i, \ldots, a_p \right) & \text{ if } 1 \leq i = j\\
		\left( 1 - \sum_{j=1}^{p} a_j, \ldots, a_{j-1}, 0, a_j, \ldots, a_p \right) &\text{ if } 0 = i < j\\
		\left( 0, \sum_{j=1}^{p} a_j, a_1, \ldots, a_p \right) &\text{ if } 0 = i = j\\
	\end{cases} 
	.\end{equation} 
	Analogously we can compute that the second term acts as
	\begin{equation}
		k_{j+1}^{p+1} \circ k_i^p (a_1, \ldots, a_p) =
	\begin{cases}
		\left( a_1, \ldots, a_{i-1}, 0, a_i, \ldots, a_{j-1}, 0, a_j, \ldots, a_p \right) & \text{ if } 1 \leq i < j\\
		\left( a_1, \ldots, a_{i-1}, 0, 0, a_i, \ldots, a_p \right) & \text{ if } 1 \leq i = j\\
		\left( 1 - \sum_{j=1}^{p} a_j, \ldots, a_{j-1}, 0, a_j, \ldots, a_p \right) &\text{ if } 0 = i < j\\
		\left( 0, \sum_{j=1}^{p} a_j, a_1, \ldots, a_p \right) &\text{ if } 0 = i = j\\
	\end{cases} 
	.\end{equation} 
\end{proof}

\begin{lem}
	$\partial \circ \partial = 0$.
\end{lem}
\begin{proof}
	We are gonna prove this result only for singular $p$ simplices, then by linearity this result will hold for arbitrary singular $p$-chains.
	If $p = 0, 1$ the result is trivial.
	Let, now, $p \geq 2$ and $\sigma$ be a singular $p$-simplex.
	By definition $\partial$ acts on $\sigma$ as
	\begin{equation}
		\partial \sigma = \sum_{i=0}^{p} (-1)^i \left( \sigma \circ k_i^{p-1} \right)
	.\end{equation} 
	Hence we can compute the double \textit{boundary} as
	\begin{equation}
	\partial \circ \partial \sigma = \sum_{i=0}^{p} \sum_{j=0}^{p-1} (-1)^{i+j} \left( \sigma \circ k_i^{p-1} \circ k_j^{p-2} \right)
	.\end{equation} 
	We can divide the sum for $i \leq j$ and $i > j$, as
	\begin{align}
		\partial \circ \partial \sigma =& \sum_{j=0}^{p-1} \sum_{i=j+1}^{p} (-1)^{i+j} \left( \sigma \circ k_i^{p-1} \circ k_j^{p-2} \right) +
	\sum_{\tilde{j}=0}^{p-1} \sum_{\tilde{i}=0}^{\tilde{j}} (-1)^{\tilde{i} + \tilde{j}} \left( \sigma \circ k_{\tilde{i}}^{p-1} \circ k_{\tilde{j}}^{p-2} \right)\\
		=& \sum_{j=0}^{p-1} \sum_{i=j+1}^{p} (-1)^{i+j} \left( \sigma \circ k_i^{p-1} \circ k_j^{p-2} \right) + 
\sum_{i=1}^{p} \sum_{j=0}^{i-1} (-1)^{i+j+1} \left( \sigma \circ k_i^{p-1} \circ k_j^{p-2} \right)\\
		=& 0
	,\end{align} 
	where, in the second sum we put $i := \tilde{j} + 1$ and $j := \tilde{i}$, and we concluded with the last line since the two sums are over all $i,j$ s.t. $i > j$ and they only differ by a sign.
\end{proof}

\begin{defn}[Singular $p$-cochain on $U$]
	Let $S^p \left(U, K\right)$, with $U \stackrel{\text{open}}{\subset} M$ and $K$ a PID as usual, be the set of functions that map a singular $p$-simplex in $U$ into an element of $K$.
	An element of $S^p \left(U, K\right)$ is called \textit{singular p-cochain} on $U$.
\end{defn}

The set $S^p \left(U, K\right)$ can be made into a $K$-module by defining the following addition and scalar multiplication:
\begin{align}
	\left( kf \right) (\sigma) &:= k \cdot f(\sigma)\\
	\left( f + g \right) (\sigma) &:= f(\sigma) + g(\sigma)
.\end{align} 
Also note that each singular $p$-cochain can be extended into a \textit{homomorphism} of $S_p(U)$ into $K$ by linearity.
This actually determines an isomorphism of $S^p \left(U, K\right)$ into the $K$-module of morphisms of $S_p(U)$ into $K$.
We will, hence, identify each element of $S^p(U,K)$ with the corresponding morphism.

\begin{defn}[Presheaf of singular $p$-cochains]
	We can define the functor $\widetilde{S}^p_K: \mathsf{Op}_M^{op} \to \mathsf{Mod}\left( K \right)$ as the functor which acts
	\begin{itemize}
		\item on the objects: $\widetilde{S}^p_K(U) := S^p \left(U, K\right)$, for any $U \in \mathsf{Op}_M$,
		\item on the morphisms: $V \stackrel{\text{open}}{\subset}  U$ iff we have $U \xrightarrow{} V$ in $\mathsf{Op}_M^{op}$. $S^p_K$ maps this morphism to $\rho_{V,U}: S^p \left(U, K\right) \to S^p \left(V, K\right)$, the function which maps any $f \in S^p \left(U, K\right)$ to its restriction to singular $p$-simplices on $V$. 
	\end{itemize} 
	Clearly, for any $p \geq 1$ these are \textit{presheaves}.
\end{defn}

Note that these \textit{presheaves} satisfy \textbf{S2}, but not \textbf{S1}:
\begin{description}
	\item[S2:] Given $\mathcal{U} := \left\{ U_i \right\}_{i \in I}$ an open covering of $U \stackrel{\text{open}}{\subset} X$ and $p$-cochains $f_i \in \widetilde{S}_K^p(U_i)$ s.t. $\left.f_i\right|_{U_i \cap U_j} = \left.f_j\right|_{U_i \cap U_j}$ for any $i, j \in I$ we can construct $f \in \widetilde{S}^p_K(U)$ s.t. $\left.f\right|_{U_i} = f_i$ for every $i \in I$.
		In fact $\left.f\right|_{U_i}$ is $f$ acting only on singular $p$-simplices in $U_i$.
		This, combined with the fact that we are not imposing any continuity conditions on $f$ means that we can take $f$ to be define as $f_i$ when computed on any singular $p$-simplex in $U_i$ and any arbitrary value -- e.g. 0 -- when computed on a singular $p$-simplex with range not contained in any $U_i$.
	\item[S1:] The last sentence should make clear the fact that the extension is not unique -- i.e. we can map the last singular simplices to any value in $K$ -- giving at least two different extensions for any set of local cochains.
		We then have that equality cannot be checked locally.
\end{description} 

This shortcoming hints to the fact that we will have to consider the \textit{sheafification} of this presheaf in order to prove an isomorphism of cohomologies (spoiler alert: we are going to use resolutions).

Let's now lay the ground for the resolution we will construct:
\begin{defn}[Coboundary homomorphism]
	One can define the \textit{coboundary homomorphism} $d(U): S^p \left(U, K\right) \to S^{p+1} \left(U, K\right)$ setting
	\begin{equation}
		d(U)f (\sigma) := f(\partial \sigma)
	,\end{equation} 
	for any $f \in S^p \left(U, K\right)$ and $\sigma$ a singular ($p+1$)-simplex with range in $U$.
\end{defn}

\begin{rem}
	Note that, since $\partial \circ \partial = 0$, also for the coboundary homomorphism, fixed $U \stackrel{\text{open}}{\subset} M$, we have $d(U) \circ d(U) = 0$.
	Moreover $d$ yields a \textit{morphism of presheaves}:
	\begin{equation}
		d: \widetilde{S}_K^p \to \widetilde{S}_K^{p+1}
	.\end{equation}
	In order to check it we have to show that, for any $V \stackrel{\text{open}}{\subset} U \in \mathsf{Op}_M$, the following diagram commutes:
	\begin{equation}
	\begin{tikzcd}
		S^p \left(U, K\right) \arrow[r, "{d(U)}", rightarrow] \arrow[d, "{\rho_{U,V}}"', rightarrow] & S^{p+1} \left(U, K\right) \arrow[d, "{\rho_{U,V}}", rightarrow] \\
		S^p \left(V, K\right) \arrow[r, "{d(V)}"', rightarrow] & S^{p+1} \left(V, K\right)
	\end{tikzcd}
	,\end{equation} 
	where, as defined above, $d(U)$ represents the \textit{coboundary} morphism
	from $S^p \left(U, K\right)$ and $d(V)$ the one from $S^p \left(V, K\right)$.
	The diagram clearly commutes, since $d$ commutes with restriction
	(the boundary operator is not affected: the range of $\sigma$ has to be in the restricted domain).
\end{rem}

\begin{defn}[Complex of presheaves of singular cochains]\label{def:precocomp}
	Consider the following chain of presheaves of \textit{singular} $q$-\textit{cochains} (consider them to be $0$ for every $q < 0$):
	\begin{equation}
		\ldots \to 0 \to \widetilde{S}^0_K \xrightarrow{d} \widetilde{S}^1_K \xrightarrow{d} \widetilde{S}^2_K \xrightarrow{d} \ldots
	.\end{equation}
	This is a complex, since we have proved that the coboundary morphism satisfies
	\begin{equation}
	d \circ d = 0
	.\end{equation}
	We denote it with $\widetilde{S}^\bullet_K$ and call it the \textit{complex of presheaves of singular cochains}.
\end{defn}

\begin{defn}[Singular cohomology]
	We associate, to the global sections of the complex of presheaves of singular cochains, the classical singular cohomology, i.e.
	\begin{equation}
		H^q_\Delta(M) := H^q(\widetilde{S}^\bullet_K(M))
	.\end{equation}
\end{defn}

Let's now create the corresponding \textit{complex of sheaves}:

\begin{defn}[Sheaf of singular $p$-cochains]
	We define the \textit{sheaf of p-cochains} $S^p_K := \big( \widetilde{S}_K^p \big)^a$ the \textit{sheafification} of the presheaf of $p$-cochains.
\end{defn}

With the aim of obtaining a $\Gamma(M;-)$-injective resolution for the constant sheaf in $K$, we want to prove that the just defined sheaf is fine:
\begin{lem}
	For any $p$ the sheaf $S^p_K$ is \textbf{fine}.
\end{lem} 
\begin{proof}
	Recall that $M$ is a \textit{paracompact} manifold.
	Hence it admits a locally finite open cover $\left\{ U_i \right\}_{i \in I}$, with associated \textit{partition of unity} $\left\{ \varphi_i \right\}_{i \in I}$.
	As proved by Warner (see \cite{warner} section 5.22) we can choose $\varphi_i$ that only assume the values $0$ and $1$, with $\mathrm{supp}\varphi_i = \overline{V_i} \subset U_i$.

	We aim to exhibit a partition of unity for the sheaf $S^p_K$.
	Our starting block will be the family $\big\{\widetilde{l}_i\big\}_{i \in I}$ of endomorphisms of the presheaf $\widetilde{S}_K^p$.
	Fixed $U \stackrel{\text{open}}{\subset} M$, $f \in \widetilde{S}_K^p(U)$ and $\sigma$ a singular $p$-simplex, then these morphisms act as:
	\begin{equation}
		\widetilde{l}_i(U) (f) (\sigma) := \varphi_i(\sigma(0)) f(\sigma)
	.\end{equation} 
	The fact that these morphisms commute with restriction is trivially true, hence $\big\{\widetilde{l}_i\big\}_{i \in I}$ is really a family of morphisms of \textit{presheaves}.

	We can, now, consider $\left\{ l_i \right\}_{i \in i}$, the family of morphisms of \textit{sheaves} associated by sheafification.
	We want to prove that this is a \textit{partition of unity} for the sheaf $S^p_K$:
	\begin{itemize}
		\item By definition $\operatorname{\mathrm{supp}}l_i := \overline{\left\{ m \in M \ \middle|\ \left( l_i \right)_m \neq 0 \right\}}$.
			We also know that the stalk of the sheafification is isomorphic to the stalk of the original presheaf, so we will check the support of $l_i$ by checking the behaviour of $\widetilde{l}_i$ at the level of stalks.
			If $m \notin \overline{V}_i$, then $\exists\, U_m$ an open neighborhood of m s.t. $U_m \cap_{} \overline{V}_i = \emptyset$. 
			Since $\overline{V}_i$ is the support of $\varphi_i$ we desume that $(\widetilde{l}_i)_m f_m = 0$ for any $f_m \in ( \widetilde{S}^p_K )_m$.
			It immediately follows that $\operatorname{\mathrm{supp}}l_i \subset \overline{V}_i \subset U_i$;
		\item The morphism $\sum_{i \in I}^{} l_i (U)$ sends $s \in S^p_K(U)$ to
			\begin{equation}
				\sum_{i \in I}^{} l_i (U) (s): m \mapsto \sum_{i \in I}^{} (l_i)_m s(m)
			.\end{equation} 
			Clearly the last term is exactly $s(m)$ by definition of $\big\{ \widetilde{l}_i \big\}_{i \in X}$.
			This shows that
			\begin{equation}
				\sum_{i \in I}^{} l_i (U) = id_{S^p_K}
			,\end{equation} 
			hence that $\left\{ l_i \right\}_{i \in i}$ is a \textit{partition of unity} for  $S^p_K$.\qedhere
	\end{itemize} 
\end{proof}


\begin{defn}[Complex of sheaves of singular cochains]
	Let us denote (with a little abuse of notation) by $d$ the image of the \textit{coboundary morphism} by the sheafification functor, $d^a$.
	Then, associated with \ref{def:precocomp}, we have the \textit{complex of sheaves of singular cochains}
	\begin{equation}
		\ldots \to 0 \to S_K^0 \xrightarrow{d} S^1_K \xrightarrow{d} S^2_K \xrightarrow{d} \ldots
	.\end{equation} 
\end{defn}

\begin{rem}
	This is going to be the starting block for the resolution of the constant sheaf $K_M$.
	Our aim in the next few lemmas is, in fact, to prove that
	\begin{equation}\label{eqn:KMResolution}
		0 \to K_M \to S^0_K \xrightarrow{d} S^1_K \xrightarrow{d} S^2_K \xrightarrow{d} \ldots
	,\end{equation} 
	is exact.
\end{rem}

\begin{rem}
	Since the functor $(-)^a$ is exact we can check exactness of \eqref{eqn:KMResolution} at $K_M$ and at $S^0_K$ by checking the exactness of
	\begin{equation}
		0 \to K_M \to \widetilde{S}^0_K \xrightarrow{d} \widetilde{S}^1_K
	.\end{equation} 
	\begin{itemize}
		\item It is exact at $K_M$ iff the morphism from the constant sheaf with values in $K$ in $\widetilde{S}^0_K$ is mono.
		It is mono iff for any $U$ open in $M$ the associated morphism of modules is injective.
		This map just sends a locally constant function $f \in K_M(U)$ to its associated locally constant singular $0$-cochain (recall that a singular $0$-simplex is just the data of a point in the manifold), hence it's clearly injective.
		\item It is exact at $\widetilde{S}^0_K$ iff $\ker d$ contains only the locally constant $0$-cochains.
			This is the case, since by definition
			\begin{equation}
				df (\sigma) := f (\partial \sigma) = f(1) - f(0)
			.\end{equation} 
			Since any singular $1$-chain is by definition a continuous path we desume that $f \in \ker d$ iff $f$ is constant on every connected subset of $M$, i.e. iff it is locally constant. 
	\end{itemize} 
\end{rem}

\begin{rem}
	For $p \geq 1$ we are going to check exactenss at the level of stalks.
	Let's fix $p$ and $m \in M$, we want to check exactness of
	\begin{equation}
		(S^{p-1}_K)_m \xrightarrow{d_m} \left( S^p_K \right)_m \xrightarrow{d_m} ( S^{p+1}_K )_m
	\end{equation} 
	at $\left( S^p_K \right)_m$.
	We have already proved that $d \circ d = 0$, hence also $d_m \circ d_m = 0$.
	This means we only have to prove that, given $f \in \left( S^p_K \right)_m$ s.t. $d_m f = 0$, there exist $g \in ( S^{p-1}_K )_m$ for which $f = dg$.

	Since we are at the level of stalks we can consider the stalks of the presheaves: $(\widetilde{S}^p_K)_m$.
	Here $d_mf = 0$ iff there exists a small enough neighborhood of $m$ s.t. $d(\left.f\right|_{U}) = 0$.
		Similarly $f = dg$ iff there exists a small enough neighborhood of $m$ on which $d(\left.g\right|_{U}) = \left.f\right|_{U}$.
\end{rem}

\begin{rem}
	Moved by the above remark we are going to concentrate on an arbitrarily small neighborhood $U$ of $m$.
	Since we are in a manifold $M$, which is locally \textit{euclidean}, we can assume $U$ to be the open unit ball in $\R^d$, where $d$ is the dimension of $M$.

	We are not going to prove exactness, but the stronger fact that the following complex
	\begin{equation}
		0 \to S^0(U,K) \xrightarrow{d} S^1(U,K) \xrightarrow{d} S^2(U,K) \xrightarrow{d} \dots
	\end{equation} 
	is homotopic to zero.
	In other words we have to construct a family of morphisms
	\begin{equation}
		h_p: S^p(U,K) \to S^{p-1}(U,K)
	,\end{equation} 
	for all $p \geq 1$, s.t.
	\begin{equation}
		d \circ h_p + h_{p+1} \circ d = id_{S^p(U,K)}
	.\end{equation} 
	In fact, if we take $f \in S^p \left(U, K\right)$ s.t. $df = 0$ then, by the above formula, we get
	\begin{align}
		f = id(f) &= (d \circ h_p + h_{p+1} \circ d) f = d \circ h_p (f)
	.\end{align} 
	If we call $g := h_p (f)$, then we obtain the desired result: the complex is exact.
\end{rem}

As just stated, for the following theorems, 
\begin{equation}
	U := \mathbb{B} = \left\{ x \in \R^d \ \middle|\ \norm{x} < 1 \right\}
,\end{equation} 
where $d$ is the dimension of $M$, i.e. the generic simply connected open neighborhood in the manifold $M$.

\begin{defn}[]
	Let $f \in S^p_K(U)$ and $\sigma$ a singular ($p-1$)-simplex in $U$.
	We define $h_p: S^p(U,K) \to S^{p-1}(U,K)$ by
	\begin{equation}
		h_p (f) (\sigma) := f\big(\widetilde{h}_p (\sigma)\big)	
	,\end{equation} 
	where $\widetilde{h}_p(\sigma)$ is the $p$-simplex that maps the origin, in $\Delta^p$, to the origin, in $U$, and, for any $(a_1, \ldots, a_p) \neq 0$, defined by
	\begin{equation}
		\widetilde{h}_p(\sigma) (a_1, \ldots, a_p) := \left( \sum_{j=1}^{p} a_j \right) \sigma \left(\frac{a_2}{\sum_{j=1}^{p} a_j}, \ldots, \frac{a_p}{\sum_{j=1}^{p} a_j} \right)
	.\end{equation}
	In what follows we will consider $\widetilde{h}_p$ to be the linear extension of the above definition to the $p$-cochains, i.e. to $\widetilde{h}_p: S_{p-1}(U) \to S_p(U)$.
\end{defn}

\begin{rem}
	With this definition we are only granting \textit{continuity} to $\widetilde{h}_p(\sigma)$, but not \textit{differentiability}.
	This is one of the few points in which the theories for continuous and differentiable simplices diverge.
\end{rem}

\begin{lem}\label{lem:boundary_hom}
	$\partial \circ \widetilde{h}_{p+1} + \widetilde{h}_p \circ \partial = id$.
\end{lem} 
\begin{proof}
	It is a simple -- and tedious -- computation:
	let $\sigma$ be a singular $p$-simplex.
	\begin{align}
		\left(\partial \circ \widetilde{h}_{p+1} + \widetilde{h}_p \circ \partial\right)(\sigma) &= \sum_{i=0}^{p+1} (-1)^i  \left( \widetilde{h}_{p+1}(\sigma) \right)^i + \sum_{i=0}^{p} (-1)^i\, \widetilde{h}_p (\sigma^i)\\
													 &= \left( \widetilde{h}_{p+1}(\sigma) \right)^0 + \sum_{i=1}^{p+1} (-1)^i \left\{ \left( \widetilde{h}_{p+1}(\sigma) \right)^i - \widetilde{h}_p(\sigma^{i-1}) \right\}\label{eqn:defn_boundary_hom}
	.\end{align} 
	Let's now compute the individual terms.

	Let $i > 1$, by definition the first summand is
	\begin{align}
		&\hphantom{=} \left( \widetilde{h}_{p+1}(\sigma) \right)^i (a_1, \ldots, a_p) = \widetilde{h}_{p+1}(\sigma) \circ k_i^p (a_1, \ldots, a_p)\\ 
		&=\ \widetilde{h}_{p+1}(\sigma)(a_1, \ldots, a_{i-1}, 0, a_{i+1}, \ldots, a_p)\\
		&= \left( \sum_{j=1}^{p} a_j \right) \sigma\left(\frac{a_2}{\sum_{j=1}^{p} a_j}, \ldots, \frac{a_{i-1}}{\sum_{j=1}^{p} a_j}, 0, \frac{a_{i+1}}{\sum_{j=1}^{p} a_j}, \ldots, \frac{a_p}{\sum_{j=1}^{p} a_j} \right)
	.\end{align} 
	Again, by definition, the second summand is
	\begin{align}
		&\hphantom{=} \left( \widetilde{h}_{p}(\sigma^{i-1}) \right) (a_1, \ldots, a_p) = \widetilde{h}_{p}\left(\sigma \circ k_{i-1}^{p-1}\right) (a_1, \ldots, a_p)\\
		&= \left( \sum_{j=1}^{p} a_j \right) \left(\sigma \circ k_{i-1}^{p-1}\right) \left(\frac{a_2}{\sum_{j=1}^{p} a_j}, \ldots, \frac{a_p}{\sum_{j=1}^{p} a_j} \right)\\
		&= \left( \sum_{j=1}^{p} a_j \right) \sigma\left(\frac{a_2}{\sum_{j=1}^{p} a_j}, \ldots, \frac{a_{i-1}}{\sum_{j=1}^{p} a_j}, 0, \frac{a_{i+1}}{\sum_{j=1}^{p} a_j}, \ldots, \frac{a_p}{\sum_{j=1}^{p} a_j} \right)
	.\end{align} 

	Let, now, $i = 1$.
	Still by definition, the first summand is
	\begin{align}
		&\hphantom{=} \left( \widetilde{h}_{p+1}(\sigma) \right)^1 (a_1, \ldots, a_p) = \widetilde{h}_{p+1}(\sigma) \circ k_1^p (a_1, \ldots, a_p)\\ 
		&=\ \widetilde{h}_{p+1}(\sigma)(0, a_1, \ldots, a_p)\\
		&= \left( \sum_{j=1}^{p} a_j \right) \sigma\left(\frac{a_1}{\sum_{j=1}^{p} a_j}, \ldots, \frac{a_p}{\sum_{j=1}^{p} a_j} \right)
	.\end{align} 
	The second summand, instead, is
	\begin{align}
		&\hphantom{=} \left( \widetilde{h}_{p}(\sigma^{0}) \right) (a_1, \ldots, a_p) = \widetilde{h}_{p}\left(\sigma \circ k_{0}^{p-1}\right) (a_1, \ldots, a_p)\\
		&= \left( \sum_{j=1}^{p} a_j \right) \left(\sigma \circ k_{0}^{p-1}\right) \left(\frac{a_2}{\sum_{j=1}^{p} a_j}, \ldots, \frac{a_p}{\sum_{j=1}^{p} a_j} \right)\\
		&= \left( \sum_{j=1}^{p} a_j \right) \sigma\left(\frac{a_1}{\sum_{j=1}^{p} a_j}, \ldots, \frac{a_p}{\sum_{j=1}^{p} a_j} \right)
	,\end{align} 
	where, in the last equality, we used
	\begin{equation}
	1 - \frac{\sum_{j=2}^{p} a_j}{\sum_{j=1}^{p} a_j} = \frac{\sum_{j=1}^{p} a_j - \sum_{j=2}^{p} a_j}{\sum_{j=1}^{p} a_j} = \frac{a_1}{\sum_{j=1}^{p} a_j}
	.\end{equation} 
	All of the abore terms are equal and, in \eqref{eqn:defn_boundary_hom}, they appear with different sign.
	This allows us to simplify the first equality to:
	\begin{equation}
		\left(\partial \circ \widetilde{h}_{p+1} + \widetilde{h}_p \circ \partial\right)(\sigma) = \left( \widetilde{h}_{p+1}(\sigma) \right)^0
	.\end{equation} 
	Let's compute this last term:
	\begin{align}
		&\hphantom{=} \left( \widetilde{h}_{p+1} (\sigma) \right)^0 (a_1, \ldots, a_p) = \widetilde{h}_{p+1}(\sigma) \circ k_0^p (a_1, \ldots, a_p)\\
		&=\ \widetilde{h}_{p+1}(\sigma) (1 - \sum_{j=1}^{p} a_j, a_1, \ldots, a_p)\\
		&= \left( 1 - \sum_{j=1}^{p} a_j + \sum_{j=1}^{p} a_j \right) \sigma \left( a1, \ldots, a_p \right)\\
		&= \sigma (a_1, \ldots, a_p)
	.\end{align}
	We, then, have the desired result!
\end{proof}

\begin{lem}
	$d \circ h_{p} + h_{p+1} \circ d = id$.
\end{lem} 
\begin{proof}
	From lemma \ref{lem:boundary_hom} we know that
	\begin{equation}
		\partial \circ \widetilde{h}_{p+1} + \widetilde{h}_p \circ \partial = id
	.\end{equation} 
	In order to use this result to prove our statement we are gonna compute the action of $d \circ h_{p} (f) + h_{p+1} \circ d (f)$, for $f \in S^p(U, M)$, on an arbitrary $p$-chain $\sigma$:
	\begin{align}
		\left( d \circ h_p + h_{p+1} \circ d \right) f(\sigma)	 &= d \circ h_p f (\sigma) + h_{p+1} \circ d f (\sigma)\\
								    &= h_p f (\partial \sigma) + d f (\widetilde{h}_{p+1} \sigma) \\
								    &= f\left(\widetilde{h}_p (\partial \sigma)\right) + f\left(\partial (\widetilde{h}_{p+1} \sigma)\right) \\
								    &= f \left( \widetilde{h}_p (\partial \sigma) + \partial(\widetilde{h}_{p+1} \sigma) \right) \\
								    &= f ( \sigma )
	.\end{align} 
\end{proof}

\begin{rem}
	As of this point we have checked everything we needed to in order to say that
	\begin{equation}
	0 \to K_M \to S^0_K \xrightarrow{d} S^1_K \xrightarrow{d} S^2_K \xrightarrow{d} \ldots
	\end{equation} 
	is a $\Gamma(M;-)$-injective resolution for $K_M$.
	In fact it is exact, by the above lemmas, and it is $\Gamma(M;-)$-injective, by \ref{lem:gammainjectives}, since each term is \textit{fine}.
\end{rem}

Our next task is to prove that the cohomology of the global sections of this resolution coincides with the singular cohomolgy on our manifold.
This will be a lengthy construction, for which we'll outline the most interesting parts, leaving the lengthy computations to \cite{warner}.

\begin{prop}\label{prop:PSh_ses}
	Let $P \in \mathsf{PSh}\left( k_M \right)$ be a presheaf that satisfies \textbf{S2}.
	Ler $S := P^a$ its \textbf{sheafification} and $\left( P(M) \right)_0$ the $K$-module defined by
	\begin{equation}
		\left( P(M) \right)_0 := \left\{ s \in P(M) \ \middle|\ s_m = 0\ \,\forall\, m \in M \right\}
	.\end{equation} 
	Then the sequence of modules
	\begin{equation}
		0 \to \left( P(M) \right)_0 \to P(M) \xrightarrow{\theta} S(M) \to 0 
	\end{equation} 
	is exact.
\end{prop} 
\begin{proof}
	We need to check exactness at each point in the sequence.
	\begin{itemize}
		\item At $\left( P(M) \right)_0$ is clear, since it is a submodule of $P(M)$ and the inclusion map is injective.
		\item At $P(M)$ we need to check that $\ker \theta = \left( P(M) \right)_0$.
			Let $s \in P(M)$, then
			\begin{equation}
				s \mapsto \bigg\{ M \xrightarrow{\theta(s)} \coprod_{m \in M} P_m \bigg\}
			,\end{equation} 
			where $\theta(s)(m) = s_m \in P_m$.
			It is clear that $s \in \ker \theta$ iff $s_m = 0$ for every $m \in M$, i.e. iff $s \in \left( P(M) \right)_0$.
		\item At $S(M)$ we need to check that $\theta$ is surjective.
			We will do it explicitly: given $t \in S(M)$ we will construct $s \in P(M)$ s.t. $t = \theta(s)$.
			Recall that an element $t$ in $S(M)$ is of the form
			\begin{equation}
				M \xrightarrow{t} \coprod_{m \in M} P_m \text{ s.t. } t(m) \in P_m \,\forall\, m \in M
			\end{equation} 
			satisfying $ \,\forall\, m \in M\ \exists\, V_m \stackrel{\text{open}}{\subset} M$ with $m \in V_m$ and $ \exists\, s \in P(V_m)$ s.t.  $s_x = t(x) \ \forall\, x \in V_m$.
			$\left\{ V_m \right\}_{m \in M}$ clearly is an open cover of $M$, which is \textit{paracompact}.
			We can extract a \textit{locally finite} open refinement $\left\{ U_\alpha \right\}_{\alpha \in \mathcal{A}}$ in which, for every $\alpha \in \mathcal{A}$, there exists $s_\alpha \in P(U_\alpha)$ s.t. $\theta(s_\alpha) = \left.t\right|_{U_\alpha}$.
				Let, now, $\left\{ V_\alpha \right\}_{\alpha \in \mathcal{A}}$ be a refinement s.t. $\overline{V}_\alpha \subset U_\alpha$.
				Let $I_m$ be the collection of indeces $\alpha$ for which $m \in \overline{V}_\alpha$ and $W_m$ a neighborhood of $m$ that satisfies:
				\begin{itemize}
					\item $W_m \cap \overline{V}_\alpha =\emptyset$ if $\alpha \notin I_m$,
					\item $W_m \subset \cap_{\alpha \in I_m} U_\alpha$, which is open and nonempty,
					\item $\left.s_\alpha\right|_{W_m} = \left.s_\beta\right|_{W_m}$ for any $\alpha, \beta \in I_m$, possible by the definition of stalk and sheafification.
				\end{itemize} 
				Let $s_m \in W_m$ be the common image of the third point.

				Consider, now, $n, m \in U$ s.t. $W_m \cap W_n \neq \emptyset$ and $p \in W_m \cap W_n$.
				From the first condition we know that $I_p \subset I_n \cap I_m$.
				Let $\alpha \in I_p$.
				By the third condition we have
				\begin{equation}
				s_m = \left.s_\alpha\right|_{W_m} \quad \text{ and } \quad \left.s_\alpha\right|_{W_n} = s_n 
				.\end{equation} 
				This imples
				\begin{equation}
				\left.s_m\right|_{W_m \cap W_n} = \left.s_\alpha\right|_{W_m\cap W_n} = \left.s_n\right|_{W_m \cap W_n}
				.\end{equation} 
				We have just constructed a family $\left\{ s_m \right\}_{m \in U}$ of sections of the presheaf that satisfy condition \textbf{S2}.
				We can hence patch them together to obtain an element $s \in P(M)$ s.t.
				\begin{equation}
					\left.s\right|_{W_m} = s_m
				\end{equation} 
				for any $m$.
				Then, by definition, $\theta(s) = t$. \qedhere
	\end{itemize} 
\end{proof}

\begin{thm}\label{thm:iso_cohomologies}
	Let $M$ be a \textbf{paracompact} manifold, $K$ a PID, then for every $q \geq 0$ 
	\begin{equation}
		H^q_\Delta(M) \simeq H^q \left( S^\bullet_K(M) \right)
	.\end{equation} 
\end{thm}
\begin{proof}
	Recall that 
	\begin{equation}
		H^q_\Delta(M) = H^q(\widetilde{S}_K^\bullet(M))
	.\end{equation} 
	This means that we have to prove
	\begin{equation}
		H^q(\widetilde{S}_K^\bullet(M))\simeq H^q \left( S^\bullet_K(M)\right)
	.\end{equation} 
	By proposition \ref{prop:PSh_ses}, applied to the \textit{presheaf} $\widetilde{S}^p_K$, we have the following short exact sequence of $K$-modules
	\begin{equation}
		0 \to (\widetilde{S}^p_K(M))_0 \to \widetilde{S}^p_K(M) \xrightarrow{\theta} S^p_K(M)
	,\end{equation} 
	for any $p \in \N$.
	It immediately follows that the associated sequence of complexes
	\begin{equation}
		0 \to (\widetilde{S}^\bullet_K(M))_0 \to \widetilde{S}^\bullet_K(M) \xrightarrow{\theta^\bullet} S^\bullet_K(M)
	\end{equation} 
	is exact.
	As usual, to this exact sequence of complexes we can associate the long cohomology sequence
	\begin{equation}
		\dots \to H^q\big((\widetilde{S}^\bullet_K(M))_0\big) \to H^q\big(\widetilde{S}^\bullet_K(M)\big) \xrightarrow{\theta^q} H^q \big(S^\bullet_K(M)\big) \xrightarrow{\delta^q} H^{q+1}\big((\widetilde{S}^\bullet_K(M))_0\big) \to \dots
	\end{equation} 
	If we managed to prove that, for every $q$, $H^q \big( (\widetilde{S}^\bullet_K(M))_0 \big) = 0$, then $\theta^q$ would be both mono and epi, hence (we are working in the category of $K$-modules) an iso.

	Let's now check it explicitly:
	\begin{description}
		\item[$\mathbf{q < 0}$:] For $q < 0$ $(\widetilde{S}^q_K(M))_0 = 0$, then also the associated cohomology module is,
		\item[$\mathbf{q = 0}$:] Note that, for $q =0$, $\widetilde{S}^0_K$ is actually a sheaf.
			In fact singular $0$-simplices in $M$ are just the assignment of a point in $M$.
			It follows that a singular $0$-cochain on $M$ is just a function from $M$ to $K$.
			If it is determined in an open cover of $M$, then it is uniquely determined, recall that $\widetilde{S}^0_K$ satisfies \textbf{S1}.
			This means that any element of $(\widetilde{S}^0_K(M))_0$ is the zero element of $\widetilde{S}^0_K(M)$.
		\item[$\mathbf{q>0}$:] We won't explicitly give the whole construction for this case, but we will reference a technical lemma, found in \cite{warner}.
			Let $\mathfrak{U} = \left\{ U_i \right\}_{i \in I}$ be an open cover of $M$.
			We define $S^\bullet_\mathfrak{U}(M,K)$ to be the set of singular cochains $f$ with values in $K$, defined only on $\mathfrak{U}$-small singular simplices.
			We say that a singular simplex is $\mathfrak{U}$-small iff its range is contained in $U_i$ for one $U_i \in \mathfrak{U}$.
			Clearly to any element $f \in S^p(M,K)$ one can associate an element in $S^p_\mathfrak{U}(M,K)$ given by the restriction of $f$ to $\mathfrak{U}$-small singular $p$-simplices only.
			This association gives rise to a surjective morphism of cochains
			\begin{equation}
				j_\mathfrak{U}: S^\bullet(M,K) \to S^\bullet_\mathfrak{U}(M,K)
			.\end{equation} 
			Denoting with $K_\mathfrak{U}^\bullet$ the kernel of this morphism, we get a short exact sequence of complexes
			\begin{equation}
				0\to K^\bullet_\mathfrak{U} \xrightarrow{} S^\bullet(M,K) \xrightarrow{j_\mathfrak{U}} S_\mathfrak{U}^\bullet(M,K) \to 0
			.\end{equation}
			From this we obtain a corresponding long exact cohomology sequence
			\begin{equation}
				\dots \to H^q(K^\bullet_\mathfrak{U}) \xrightarrow{i_\mathfrak{U}^q}  H^q(S^\bullet(M,K)) \xrightarrow{j_\mathfrak{U}^q}
				H^q(S^\bullet_\mathfrak{U}(M,K)) \xrightarrow{\delta^q} H^{q+1}(K^\bullet_\mathfrak{U}) \to \dots
			.\end{equation} 
			We want to prove that the map $j_\mathfrak{U}^q$ induced in the cohomology sequence is an iso for any $q$.
			If it were the case, it would follow that, for every $q$
			\begin{equation}\label{eqn:0cohomology}
				H^q(K^\bullet_\mathfrak{U}) = 0
			.\end{equation} 
			In fact, since $j_\mathfrak{U}^q$ induces an iso, it means that $\ker \delta^q = H^q(S^\bullet(M,K))$ and $\ima \delta^q = 0$. 
			But $\ima \delta^q = \ker i_\mathfrak{U}^q$, i.e. this last map is a mono.
			Analogously it can be proved that $\ima i^q_\mathfrak{U} = 0$, hence that $H^q(K^\bullet_\mathfrak{U}) = 0$.

			Following \cite{warner} this statement is proved using the results of lemma \ref{lem:longlemma}, below.
			In fact, by \eqref{eqn:surjectivity}, $j_\mathfrak{U}$ induces surjections of the cohomology modules.
			By \eqref{eqn:injectivity} $k \circ j_\mathfrak{U}$ induces the identity on cohomology, which means that $j_\mathfrak{U}$ must induce injections.
			Putting it all together $j_\mathfrak{U}$ induces isomorphisms.

			With this fact we want to prove that $H^q \big((\widetilde{S}^\bullet_K(M))_0\big) = 0$.
			Consider $f \in (\widetilde{S}^q_K(M))_0$ s.t. $df = 0$.
			By definition of $(\widetilde{S}^q_K(M))_0$ we know that $f_m = 0$ for every $m \in M$, hence that for any  $m$ there exist an open neighborhood $U_m$ of $m$ s.t. $f$ maps every singular $q$-chain with range in $U_m$ to $0$.
			This means that there exists an open cover $\mathfrak{U}$ of $U$ consisting of sufficiently small sets, for which $f \in K^q_\mathfrak{U}$.
			From \eqref{eqn:0cohomology} we know that $\exists\, g \in K^{q-1}_\mathfrak{U} \subset (\widetilde{S}^{q-1}_K(M))_0$ s.t. $f = dg$, i.e. 
			\begin{equation}
				H^q \big((\widetilde{S}^\bullet_K(M))_0\big) = 0
			.\end{equation}
	\end{description} 
\end{proof}

\begin{lem}\label{lem:longlemma}
	Let $M$ be a \textbf{paracompact} manifold,
	\begin{equation}
		j_\mathfrak{U}: S^\bullet(m,K) \to S^\bullet_\mathfrak{U}(M,K)	
	\end{equation} 
	be the inclusion map defined in the above proof.
	Then there exists a map 
	\begin{equation}\label{eqn:surjectivity}
		k: S^\bullet_\mathfrak{U}(M,K) \to S^\bullet(M,K)	
	\end{equation} 
	s.t. $j_\mathfrak{U} \circ k = id$ and homotopy operators $h^p: S^p(M,K) \to S^{p-1}(M,K)$ s.t.
	\begin{equation}\label{eqn:injectivity}
		h^{p+1} \circ d + d \circ h^p = id - k^p \circ j^p
	.\end{equation} 
\end{lem} 
The proof for this lemma consists in explicitly constructing these maps.
It is rather long and boring.
Also we have used almost the same notation as \cite{warner}, so it can be checked there -- in section 5.32 -- without any issue.

Now, before moving onto the main result of this work, let's define a last concept:
\begin{defn}[Integration of forms over singular simplices]
	Let $M$ be a \textit{differentiable manifold}, $\sigma$ a \textit{differentiable} singular $p$-simplex in $U$ and $\omega$ a \textit{continuous} $p$-form also defined on $U \stackrel{\text{open}}{\subset} M$.
	\begin{itemize}
		\item If $p = 0$ $\sigma$ is just the data of a point, $\sigma(0)$, in $U$ and $\omega$ just a continuous (differentiable) function.
			We define the integral of $\omega$ over $\sigma$ to be
			\begin{equation}
				\int_{\sigma}^{} \omega := \omega(\sigma(0)) 
			.\end{equation} 
		\item If $p \geq 1$ $\sigma$ extends to a smooth map from a neighborhood of $\Delta^p$ into $U$.
			This implies that the pullaback of $\omega$ is defined in a neighborhood of $\Delta^p$.
			We can then compute its integral on the $p$-simplex.
			From this we define the integral of $\omega$ over $\sigma$ to be
			\begin{equation}
				\int_{\sigma}^{} \omega := \int_{\Delta^p}^{} \sigma^*(\omega)  
			.\end{equation} 
	\end{itemize} 
	We linearly extend those definitions to chains:
	let $c = \sum_{i}^{}a_i \sigma_i$, then
	\begin{equation}
	\int_{c}^{} \omega := \sum_{i}^{} a_i \int_{\sigma_i}^{} \omega  
	.\end{equation} 
\end{defn}
Let's now state (without proving it) a famous theorem that links the \textit{exterior differential} for forms, with the \textit{boundary operator} for simplices.

\begin{thm}[Stokes' theorem]
	Let $M$ be a differentiable manifold, $U \stackrel{\text{open}}{\subset} M$, $c$ a $p$-chain in $U$, with $p \geq 1$, and $\omega$ be a smooth $(p-1)$-form, still defined on $U$.
	Then
	\begin{equation}
	\int_{\partial c}^{} \omega = \int_{c}^{} d\omega
	.\end{equation} 
\end{thm}
If you are interested in the proof you can find it in \cite[\S 4.7]{warner}.

\begin{rem}
	Let us define, for $p \geq 0$, the following homomorphism
	\begin{equation}
		k^p: \Omega^p(U) \to S^p(U,\R)
	\end{equation} 
	by setting, for $\omega \in \Omega^p(U)$ and $\sigma$ a differentiable singular $p$-simplex,
	\begin{equation}
		k^p(\omega)(\sigma) := \int_{\sigma}^{} \omega 
	.\end{equation} 
	Then \textit{Stokes' theorem} makes this a morphism of complexes
	\begin{equation}
		k^\bullet: \Omega^\bullet(U) \to \widetilde{S}^\bullet_\R(U)
	.\end{equation} 
	We, in fact, need to check the commutativity of the following diagram
	\begin{equation}
	\begin{tikzcd}
		\Omega^p(U) \arrow[r, "d", rightarrow] \arrow[d, "k^p"', rightarrow] & \Omega^{p+1}(U) \arrow[d, "k^{p+1}", rightarrow] \\
		S^p(U,\R) \arrow[r, "d", rightarrow] & S^{p+1}(U,\R)
	\end{tikzcd}
	,\end{equation} 
	i.e. that $k^{p+1} \circ d = d \circ k^p$.
	Let $\sigma$ and $\omega$ be as above, then
	\begin{align}
		(d \circ k^p)(\omega) (\sigma) &= k^p(\omega) (\partial\sigma) = 
		\int_{\partial\sigma}^{} \omega 
		= \int_{\sigma}^{} d\omega \\
		&= k^{p+1}(d\omega) (\sigma) = (k^{p+1} \circ d) (\omega) (\sigma)
	.\end{align} 
	This, in turn, gives a morphism of the respective cohomologies:
	\begin{equation}
		k^q: H^q(\Omega^\bullet(U)) \to H^q(S^\bullet(U,\R))
	,\end{equation} 
	called the \textit{de Rham homomorphism}.
	Moreover, since this family of morphisms clearly commutes with restrictions, it also induces a family of morphisms of \textit{presheaves}:
	\begin{equation}
		k^p: \Omega^p \to \widetilde{S}^p_\R
	.\end{equation} 
	Which, as before, gives a morphism of complexes of presheaves
	\begin{equation}
		k^\bullet: \Omega^\bullet \to \widetilde{S}^\bullet_\R
	.\end{equation} 
\end{rem}

We are finally ready to state and prove our final result:
\begin{thm}[De Rham]
	Let $M$ be a \textit{smooth paracompact manifold}.
	Then, for every $q$, there exists an isomorphism $H^q_{DR}(M) \simeq H^q_\Delta(M)$.
	Moreover this isomorphism is given by integration of forms on singular chains.
\end{thm}
\begin{proof}
	When taking the PID $K = \R$ to be the field of real numbers, we have constructed two $\Gamma(M,-)$-injective resolutions for the sheaf of constant functions in $\R$:
	\begin{equation}
		0 \to \R_M \to \Omega^0 \xrightarrow{d} \Omega^1 \xrightarrow{d} \Omega^2 \xrightarrow{d} \dots
	,\end{equation} 
	and
	\begin{equation}
		0 \to \R_M \to S^0_\R \xrightarrow{d} S^1_\R \xrightarrow{d} S^2_\R \xrightarrow{d} \dots
	.\end{equation} 
	Moreover we know that $\mathsf{Mod}\left( \R_M \right)$ has enough \textit{injectives} and $\Gamma(M,-)$ is left exact.
	It follows that we can compute the $q$-th right derived functor of $\Gamma(M,-)$ by computing the $(q-1)$-th cohomology of these two sequences.
	As a consequence we have the following isomorphism
	\begin{equation}\label{eqn:cohom_iso_res}
		H^q_{DR}(M) \simeq H^q(S^\bullet_\R(M))
	.\end{equation} 
	By theorem \ref{thm:iso_cohomologies} we know that $H^q_\Delta(M) \simeq H^q \left( S^\bullet_\R(M)\right)$, which gives the desired isomorphism
	\begin{equation}
		H^q_{DR}(M) \simeq H^q_\Delta(M)
	.\end{equation} 
	We are only left to prove that this isomorphism is the above defined \textit{de Rham isomorphism}.
	In fact it gives rise to the following morphism of complexes
	\begin{equation}
	\begin{tikzcd}
		0 \arrow[r, "", rightarrow] & \R_M \arrow[r, "", rightarrow] \arrow[d, "id", rightarrow] & \Omega^0 \arrow[r, "d", rightarrow] \arrow[d, "k^0", rightarrow] & \Omega^1 \arrow[r, "d", rightarrow] \arrow[d, "k^1", rightarrow] & \dots \\
		0 \arrow[r, "", rightarrow] & \R_M \arrow[r, "", rightarrow] & \widetilde{S}^0_\R \arrow[r, "d", rightarrow] & \widetilde{S}^1_\R \arrow[r, "d", rightarrow] & \dots
	\end{tikzcd}
	.\end{equation} 
	This, in turn, by \textit{sheafification} becomes
	\begin{equation}
	\begin{tikzcd}
		0 \arrow[r, "", rightarrow] & \R_M \arrow[r, "", rightarrow] \arrow[d, "id", rightarrow] & \Omega^\bullet \arrow[d, "{(k^\bullet)^a}", rightarrow] \\
		0 \arrow[r, "", rightarrow] & \R_M \arrow[r, "", rightarrow] & S^\bullet_\R
	\end{tikzcd}
	.\end{equation} 
	Since both of those are $\Gamma(M,-)$-injective resolutions, $(k^\bullet)^a$ induces a quasi isomorphism, i.e. isomoprhisms at the level of cohomologies, between the following complexes
	\begin{equation}
	\begin{tikzcd}
		0 \arrow[r, "", rightarrow] & \R_M \arrow[r, "", rightarrow] \arrow[d, "id", rightarrow] & \Omega^0(M) \arrow[r, "d", rightarrow] \arrow[d, "{(k^0)^a}", rightarrow] & \Omega^1(M) \arrow[r, "d", rightarrow] \arrow[d, "{(k^1)^a}", rightarrow] & \dots \\
		0 \arrow[r, "", rightarrow] & \R_M \arrow[r, "", rightarrow] & S^0_\R(M) \arrow[r, "d", rightarrow] & S^1_\R(M) \arrow[r, "d", rightarrow] & \dots
	\end{tikzcd}
	.\end{equation} 
	Moreover we have proved that $F$-injective resolutions differ by a unique isomorphism, at the level of cohomology, hence the canonical isomorphism in equation \eqref{eqn:cohom_iso_res} corresponds with the one induced by $(k^\bullet)^a$.
	This, in turn, means that the following commutative diagram
	\begin{equation}
	\begin{tikzcd}
		\Omega^\bullet(M) \arrow[d, "k^\bullet"', rightarrow] \arrow[rd, "(k^\bullet)^a", rightarrow] & \\
		\widetilde{S}^\bullet_\R(M) \arrow[r, "\theta"', rightarrow] & S^\bullet_\R(M)
	\end{tikzcd}
	,\end{equation} 
	where $\theta$ is the natural morphism $F \xrightarrow{\theta} F^a$, gives rise to the following commutative diagram of cohomologies
	\begin{equation}
	\begin{tikzcd}
		H^\bullet_{DR}(M) \arrow[d, "k^\bullet"', rightarrow] \arrow[rd, "\sim", rightarrow] & \\
		H^\bullet_\Delta(M) \arrow[r, "\sim", rightarrow] & H^\bullet(S^\bullet_\R(M))
	\end{tikzcd}
	.\end{equation} 
	Since the other two are isomorphisms, also $k^\bullet$ is.
	Moreover it is the canonical (and unique) one, we have found before.
\end{proof}
